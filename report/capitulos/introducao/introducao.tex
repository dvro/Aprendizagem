
\chapter{Introdução}

Este capítulo dará uma breve introdução sobre este trabalho e os objetivos do mesmo.

\section{Motivação}

Hoje, existem diversos algoritmos de classificação com diferentes características, cada um deles apropriada para um tipo de estrutura de dados. Este trabalho tem como objetivo abordar conceitos de clusterização e classificação, assim como expor os principais algoritmos de classificação e clusterização.

Por fim, este trabalho também tem por objetivo analisar e comparar o desempenho dos dados classificadores, utilizando uma mesma base artificial, e tendo como métricas a taxa de erro global e por classes. Para comparação, foi utilizada a verificação da hipótese nula pelo teste T de Student.


\section{Estrutura do Trabalho}
Este trabalho inicia com um breve histórico, seguindo com ideias gerais sobre algoritmos de agrupamento e o K-Means. Logo depois, segue conceitos gerais de classificadores e detalhes a respeito do KNN, Estimação de Verossimilhança e Expectation-Maximization, seguindo por uma seção a respeito de combinação de classificadores.

No capítulo de experimentos, estão demonstrados os experimentos realizados com todas as técnicas abordadas, seguida por uma seção de comparação de equivalência de classificadores.

Por fim, segue-se a conclusão e bibliografia.

%%% Local Variables: 
%%% mode: latex
%%% TeX-master: "../mainrep"
%%% End: 
