% !TEX encoding = ISO-8859-1
\chapter{Conclusão}
\label{ch:conclusao}

Com os resultados observados, pode-se concluir que o ASGP e o ASGPM possuam uma redução de instâncias intermediária, entre o SGP1 e o SGP2.

Quanto a taxa de acerto, percebe-se que as adaptações propostas ocasionam uma pequena queda na taxa de acerto da classe marjoritária, ocasionando uma queda também na taxa de acerto geral. Porém, observa-se que, na grande maioria dos casos, a percentagem de acerto ganha na classe minoritária é muito superior a percentagem perdida na classe marjoritária.

Assim, conclui-se que o ASGP e o ASGPM são técnicas eficientes quando se deseja priorizar a classe minoritária, e é necessário reduzir drasticamente o conjunto de instâncias.

Para trabalhos futuros, propõe-se que sejam avaliadas novas formas de utilizar o $Merge$ e o $Pruning$, para encontrar um equilibrio entre a remoção de instâncias da classe marjoritária e a representação da classe minoritária.



