% !TEX encoding = ISO-8859-1


Para avaliar os classificadores, utilizou-se \textit{Holdout Cross-Validation} estratificado e repetido. A base dados foi dividida em treino e teste, utilizando a proporção 70\% e 30\%, respectivamente.

O experimento foi executado 50 vezes, para se obter métricas confiaveis de avaliação das técnicas, e para, posteriormente, utilizar o teste de hipótese.

Para avaliar o desempenho de cada classificador, foi calculada a taxa de erro de classificação global e por classes. Para o K-means, também foi calculado o Índice de Rand Corrigido.

\begin{equation}
\label{eq:eq1}
P(\omega_{i} | x_{k},\theta{i}) = \dfrac{p(x_k| \omega_i, \theta_i) \times P(\omega_i)}{\sum_{j=1}^c p(x_k | \omega_j, \theta_j) \times P(\omega_j)}
\end{equation}

\begin{equation}
\label{eq:eq2}
j = \operatorname*{arg\,max}_i P(\omega_{i} | x_{k},\theta{i})
\end{equation}

Os classificadores se baseiam na estimativa da probabilidade a posteriori das classes, especificada na equação \ref{eq:eq1}. Sendo a regra de decisão classificar um padrão como sendo da classe de maior probabilidade a posteriori, conforme equação \ref{eq:eq2}.


