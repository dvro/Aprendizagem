% !TEX encoding = ISO-8859-1


A geração de dados está demonstrada na seção \ref{sec:geracaodedados}, o K-means está abordado na seção \ref{subsec:exp-kmeans}, seção seguinte.

Nas seções \ref{subsec:exp-mle-em}, \ref{subsec:exp-janeladeparzen} e \ref{subsec:exp-combinacaodeclassificadores} estão abordados a Máxima Verossimilhança, Janela de Parzen e Combinação de Classificadores, respectivamente.

Para avaliar o desempenho de cada técnica, foi calculada a taxa de erro de classificação global e por classes. Além disso, também foi calculado o Índice de Rand Corrigido.

Para avaliação dos classificadores, foram utilizados os dados citados na seção \ref{sec:geracaodados}. Os classificadores obedecem a regra de decisão especificada abaixo:

\begin{equation}
j = argmax_i P(\omega_{i} | x_{k},\theta{i})
\end{equation}
com
\begin{equation}
P(\omega_{i} | x_{k},\theta{i}) = \dfrac{p(x_k| \omega_i, \theta_i) \times P(\omega_i)}{\sum_{j=1}^c p(x_k | \omega_j, \theta_j) \times P(\omega_j)}
\end{equation}

